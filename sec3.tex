%%%%%%%%%%%%%%%%%%%%%%%%%%%%%%%%%%%%%%%%%%%%%%%%%%%%%%%%%%%%%%%%%%%%
% Diskussion und Ausblick
%%%%%%%%%%%%%%%%%%%%%%%%%%%%%%%%%%%%%%%%%%%%%%%%%%%%%%%%%%%%%%%%%%%%

\chapter{Applications}
  \label{Applications}

\section{Trello API wrapper}

\section{Trello framework}

\section{Export to HTML}

\subsection{Twitter Bootstrap Framework}

\subsection{HTML 5}

\subsection{CSS 3 / SASS}

\subsection{ERB / Templating}

\section{One way sny to Google Calendar}

\section{Export to iCal}

\section{One way sync to Joomla}

\subsection{For every card an article}

\subsection{All cards in one article}

\subsection{One way sny to WordPress}

\section{Backup}

\subsection{Export}

\subsection{Import}

\subsubsection{Filename option}
The -n (or -name) argument for this script stands for the filename of the backup file which contains the  exported Trello data. With -n the user can specify a file to import. While processing the script first checks if the user has passed this argument. If not, it aborts. If the -n argument is given, the scipt proofes if the file is a ZIP file. For that it soesn't use the filename but the MIME type of the file.

\begin{lstlisting}[float=htb, caption=Bewegungsdaten auslesen \cite{apple:003}, label=listing008]
if `file -Ib #{@filename}`.gsub(/;.*\n/, "") != "application/zip"
	puts "ERROR: The backup file has to be a ZIP file!"
	abort
end
\end{lstlisting}

	
In line 1 the \texttt{file -Ib \#\{\@filename\}} is a bash call for receiving the MIME type of a file. Ruby executes it and with the gsub-Method it cuts the MIME part out of the received string.

\subsection{Member import}

