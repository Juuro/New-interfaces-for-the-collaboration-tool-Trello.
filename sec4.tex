%%%%%%%%%%%%%%%%%%%%%%%%%%%%%%%%%%%%%%%%%%%%%%%%%%%%%%%%%%%%%%%%%%%%
% Diskussion und Ausblick
%%%%%%%%%%%%%%%%%%%%%%%%%%%%%%%%%%%%%%%%%%%%%%%%%%%%%%%%%%%%%%%%%%%%
\onehalfspacing
\chapter{Conclusion}
  \label{Conclusion}

In this thesis a Ruby-based\index{Ruby} API\index{API} wrapper for the collaboration tool\index{collaboration tool} Trello was developed. Chapter \ref{apiwrapper} discussed its implementation and extensions to provide a convenient tool for developers to access Trello. With this API wrapper, developers can build every kind of script that works with data in Trello. A command-line interface was applied to the scripts to enable the developer to use the same script for multiple calls. Example scripts which use the API wrapper were described in chapter \ref{applications}. The first script exports Trello cards to an HTML\index{HTML} file. Thus every kind of structured data file could be created with this approach. Two other scripts extract due dates of cards in Trello to Google Calendar\index{Google!Calendar} or provide it as an iCalendar\index{iCalendar} file. To fill a CMS\index{CMS} with content stored in Trello, a synchronisation script for the popular Joomla\index{Joomla} CMS\index{CMS} was developed. Finally, a backup solution is described to save all the information of a Trello account with organizations, boards, lists, cards, and its additions to a single file. This backup file is saved as a ZIP file and includes pure JSON. Thus it is easy to read for third party applications. The backup file can also be imported again in Trello. 

Overall, this thesis provides instruments to build applications in Ruby\index{Ruby} which interact with the Trello API\index{API}. Several practical examples were presented. Most of them will be used in a productive environment.