%%%%%%%%%%%%%%%%%%%%%%%%%%%%%%%%%%%%%%%%%%%%%%%%%%%%%%%%%%%%%%%%%%%%
% Grundlagen
%%%%%%%%%%%%%%%%%%%%%%%%%%%%%%%%%%%%%%%%%%%%%%%%%%%%%%%%%%%%%%%%%%%%

\chapter{Principles}
  \label{Principles}

\section{Trello}

\subsection{How Trello works}
Trello is a webservice by the New York City based web corporation Fog Creek Software. It is a collaboration tool where you can manage your projects. There is the concept of so called \emph{boards} which contents several configurable lists. In these lists you can create todo items you're working on, these are called \emph{cards}. You can add your co-workers to these boards and cards. So everone who's working on a project can see whats going on at the moment.

\subsection{Why Trello}
Trello is not just one of hundreds of thousands of todo applications. It is streamlined for the purposes of small businesses. So for our needs in the university with small groups of people working on the same things it was perfect. Trello has proofed its value several months already.

The first wish was to see the due dates of the cards one is assigned to in Google calendar. Because Google calendar is the calendar tool of our choice. But thinking about that there were many other use cases for small scripts which could run as a cron job on a server to serve several regular tasks.

\subsection{Trello API}
Trello has an API which is still in beta at the moment I'm writing this. But it is already very extensive. 

JSON

OAuth2

Rest

\section{Ruby}

\subsection{Ruby concepts}

\subsection{Ruby Gems and packages}

\section{JSON}