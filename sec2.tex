%%%%%%%%%%%%%%%%%%%%%%%%%%%%%%%%%%%%%%%%%%%%%%%%%%%%%%%%%%%%%%%%%%%%
% Grundlagen
%%%%%%%%%%%%%%%%%%%%%%%%%%%%%%%%%%%%%%%%%%%%%%%%%%%%%%%%%%%%%%%%%%%%

\chapter{Principles}
  \label{MatMet}

\noindent
Ziel dieses Kapitels ist eine Einf\"uhrung in die Thematik BlaBlaBla ...

\section{Trello}

\subsection{How Trello works}

Trello is a webservice by the New York City based web corporation BLAHA!. It is a collaboration tool where you can manage your projects. There is the concept of boards which contents several configurable lists. In these lists you can create todo items you're working on, these are called "cards". You can add your co-workers to these boards and cards. So everone who's working on a project can see whats going on at the moment.

\subsection{Why Trello}
Trello is not just one of hundreds of thousands of todo applications. It is streamlined for the purposes of small businesses. So for our needs in the university with small groups of people working on the same things it was perfect. Trello has proofed its value several months already. But although it works fine like it's supposed to it has his boarders.

The first wish was to see the due dates of the cards one is assigned to in Google calendar. Because Google calendar is the calendar tool of our choice.

\subsection{Trello API}

\section{Ruby}

\subsection{Ruby concepts}

\subsection{Ruby Gems and packages}

\section{JSON}