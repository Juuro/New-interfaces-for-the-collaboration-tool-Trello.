%%%%%%%%%%%%%%%%%%%%%%%%%%%%%%%%%%%%%%%%%%%%%%%%%%%%%%%%%%%%%%%%%%%%
% Grundlagen
%%%%%%%%%%%%%%%%%%%%%%%%%%%%%%%%%%%%%%%%%%%%%%%%%%%%%%%%%%%%%%%%%%%%

\chapter{Principles}
  \label{Principles}

\section{Trello}

\subsection{How Trello works}
Trello is a webservice by the New York City based web corporation Fog Creek Software\footnote{\url{http://www.fogcreek.com}}\index{Fog Creek Software}. It is a collaboration tool where you can manage your projects. There is the concept of so called \emph{boards} which contents several configurable lists. In these lists you can create todo items you're working on, these are called \emph{cards}. You can add your co-workers to these boards and cards. So everone who's working on a project can see whats going on at the moment.

\subsection{Why Trello}
Trello is not just one of hundreds of thousands of todo applications. It is streamlined for the purposes of small businesses. So for our needs in the university with small groups of people working on the same things it was perfect. Trello has proofed its value several months already.

The first wish was to see the due dates of the cards one is assigned to in Google calendar. Because Google calendar is the calendar tool of our choice. But thinking about that there were many other use cases for small scripts which could run as a cron job on a server to serve several regular tasks.

\subsection{Trello API}
Trello has an API \nomenclature{API}{Application Programming Interface} which is still in beta at the moment I'm writing this. But it is already very extensive. \cite{trello:docu}

\subsubsection{REST}\nomenclature{REST}{Representational State Transfer}
The Trello API is a \emph{RESTful} web API. That means that the API is conform to the REST design model. REST is a common style of software architecture for distributed systems. An implementation of a REST web service follows four basic design principles:
\begin{itemize}
	\item Use HTTP\nomenclature{HTTP}{Hyper Text Transfer Protocol} methods explicitly.
	\item Be stateless.
	\item Expose directory structure-like URIs\nomenclature{URI}{Uniform resource identifier}.
	\item Transfer XML\nomenclature{XML}{Extensible Markup Language}, JSON\nomenclature{JSON}{JavaScript Object Notation}, or both.
\end{itemize}

\cite{rest}

\subsubsection{Authentification}
Though the scripts which are used here need access to private boards in Trello there has to be any kind of authentification. For user applications with a frontend the Trello API provides OAuth2. But because of the concept of OAuth2 the user is required to enter his Trello username and password. \cite{oauth} My scripts are supposed to run on servers as cron jobs. There is no user who could manually enter data. For this kind of applications Trello provides a key/token-system. Every user has a private key. Whith this key the user can generate a token. This token will be send along every request to the Trello API. The token tells Trello which scope the request can see. While generating a token one can specify the scope of the token and when it will expire. The possible expriations of a token are between one day and never. In our case we will use \emph{never}. To generate a token one has to visit a special URL:
\texttt{
https://trello.com/1/authorize?key=SUBSTITUTEWITHYOURPRIVATEKEY \&name=My+Application\&expiration=never\&response\_type=token \&scope=read,write}
In this example the token would never expire and could read and write everything the user can access with the API. Other valid values instead of \texttt{never} for expiration would be \texttt{1day}, \texttt{30days}. \texttt{30days} is the default value.

\section{JSON}
All the responses to Trello API calls use JSON. JSON means \emph{Javascript Object Notation}. It's not directly related to JavaScript, but it was first developed for the use with JavaScript. \todo{verify} JSON an easy markup language like XML. But JSON consists of only two conecpts: arrays and hashes. An array is a list of values. A hash is a list of key-value pairs. Both can be arbitrary nested. At every point one of my script saves content at any other place than Trello it's in the JSON format, too. That's because it guarantees easy compatibility with Trello. 
\todo{JSON Wikipedia}

\section{Ruby}
Ruby is a modern scripting language. It's very similar to Pathon and fills the same purposes as PHP, which is very popular since years. The big difference to older scripting languages like PHP is, that it's much more easier to read. Ruby doesn't require brakets. One can write \texttt{do} and \texttt{end} instead. Thats much more understandable than \texttt{\{} and \texttt{\}} like they are used in PHP. But \texttt{do} and \texttt{end} are replacable by \texttt{\{} and \texttt{\}}.

\subsection{Ruby concepts}
Ruby has a good amount of methods and classes every Ruby installation provides. But there are hundreds of extensions for special use cases – to communicate with RESTful Web APIs for example. There are two different kinds of extensions: \emph{packeges} and \emph{gems}. Ruby gems are small plugins for Ruby which provide additional methods and classes. Gems can be added to an existing Ruby installation at any time. Some Ruby distributions are delivered with several gems. \todo{Wie heißen die vorinstallierten packages?}
Ruby packages are ...