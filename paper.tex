%%%%%%%%%%%%%%%%%%%%%%%%%%%%%%%%%%%%%%%%%%%%%%%%%%%%%%%%%%%%%%%%%%%%%%%%%%%%%
%%% LaTeX-Rahmen fuer das Erstellen von Diplomarbeiten
%%%%%%%%%%%%%%%%%%%%%%%%%%%%%%%%%%%%%%%%%%%%%%%%%%%%%%%%%%%%%%%%%%%%%%%%%%%%%

%%%%%%%%%%%%%%%%%%%%%%%%%%%%%%%%%%%%%%%%%%%%%%%%%%%%%%%%%%%%%%%%%%%%%%%%%%%%%
%%% allgemeine Einstellungen
%%%%%%%%%%%%%%%%%%%%%%%%%%%%%%%%%%%%%%%%%%%%%%%%%%%%%%%%%%%%%%%%%%%%%%%%%%%%%

\documentclass[twoside,12pt,a4paper, parskip=full-]{report}
%\usepackage{reportpage}
\usepackage{epsf}
\usepackage{graphics, graphicx}
\usepackage{latexsym}
\usepackage{amsfonts}
\usepackage{amsmath}
\usepackage{listings}
\usepackage{url}
\usepackage{textcomp}
\usepackage[numbers]{natbib}
\usepackage[margin=10pt,font=small,labelfont=bf]{caption}
\usepackage[utf8]{inputenc}
\usepackage{nomencl} 
\usepackage{makeidx}
\makenomenclature
\makeindex

\usepackage{setspace}

\setlength{\parindent}{0mm}

\renewcommand*{\lstlistlistingname}{List of Listings}

\usepackage{xcolor}
\usepackage{hyperref}
\definecolor{darkred}{rgb}{0.5,0,0}
\definecolor{red}{rgb}{1,0,0}
\definecolor{darkgreen}{rgb}{0,0.5,0}
\definecolor{darkblue}{rgb}{0,0,0.5}
\definecolor{gray}{gray}{.6}

\definecolor{javared}{rgb}{0.6,0,0} % for strings
\definecolor{javagreen}{rgb}{0.25,0.5,0.35} % comments
\definecolor{javapurple}{rgb}{0.5,0,0.35} % keywords
\definecolor{javadocblue}{rgb}{0.25,0.35,0.75} % javadoc
\definecolor{softgray}{rgb}{0.96,0.96,0.96}

\newcommand{\todo}[1]{\textbf{\textsc{\textcolor{red}{TODO: #1}}}}

%\lstloadlanguages{Java,XML,HTML,C++,SQL,C}
\hypersetup{colorlinks,
linkcolor=darkblue,
citecolor=darkgreen,
urlcolor=blue
}

%\lstdefinelanguage[GNU99]{Ruby}

\lstdefinestyle{Bash}
{language=bash,
keywordstyle=\color{blue},
basicstyle=\ttfamily\scriptsize\small,
morekeywords={peter@kbpet},
deletekeywords={if},
alsoletter={:~$},
morekeywords=[2]{peter@kbpet:},
keywordstyle=[2]{\color{red}},
literate={\$}{{\textcolor{red}{\$}}}1 
         {:}{{\textcolor{red}{:}}}1
         {~}{{\textcolor{red}{\textasciitilde}}}1,
}


\lstdefinestyle{Html}{
 language=HTML,
        breaklines=true,
        commentstyle=\textit,
        keywordstyle=\bfseries,
        basicstyle=\ttfamily,
        stringstyle=\ttfamily,
        showstringspaces=false,
        frame=single,
        tabsize=2
}


\lstloadlanguages{Ruby} 
 \lstset{ 
  language=Ruby,
  numbers=left, 
  numberstyle=\color{gray}, 
  numbersep=8pt, 
  tabsize=2, 
  breaklines=true, 
  frame=single, 
  basicstyle=\ttfamily\scriptsize\small, 
  keywordstyle=\bfseries\itshape\color{javadocblue}, 
  morekeywords={puts, loop, defined, lambda, where}, 
  commentstyle=\color{gray}, 
  stringstyle=\color{javared}, 
  showspaces=false, 
  showstringspaces=false, 
  backgroundcolor=\color{softgray}, 
  title=\lstname ,
  captionpos=b
  }
\lstset{escapeinside={(*@}{@*)}}
  

\textwidth 14cm
\textheight 22cm
\topmargin 0.0cm
\evensidemargin 1cm
\oddsidemargin 1cm
%\footskip 2cm

% Kann von Student auch nach pers\"onlichem Geschmack ver\"andert werden.
\pagestyle{headings}

\sloppy

\begin{document}


\setlength{\parskip}{1ex plus 0.5ex minus 0.2ex}}

%%%%%%%%%%%%%%%%%%%%%%%%%%%%%%%%%%%%%%%%%%%%%%%%%%%%%%%%%%%%%%%%%%%%%%%%%%%%
%%% hier steht die neue Titelseite 
%%%%%%%%%%%%%%%%%%%%%%%%%%%%%%%%%%%%%%%%%%%%%%%%%%%%%%%%%%%%%%%%%%%%%%%%%%%%
 
\begin{titlepage}
 \begin{center}
  {\LARGE Eberhard Karls Universit\"at T\"ubingen}\\
  {\large Mathematisch-Naturwissenschaftliche Fakultät \\
Wilhelm-Schickard-Institut f\"ur Informatik\\[4cm]}
  {\huge Diplom Thesis Informatics\\[2cm]}
  \todo{?}
  {\Large\bf  Framework for connecting Trello to other services and applications\\[1.5cm]}
 {\large Sebastian Engel}\\[0.5cm]
11th September 2012\\[4cm]
{\small\bf Referees}\\[0.5cm]
  \parbox{7cm}{\begin{center}{\large Prof. Dr. Torsten Grust}\\
   (Informatik)\\
  {\footnotesize Wilhelm-Schickard-Institut f\"ur Informatik\\
	Universit\"at T\"ubingen}\end{center}}\hfill\parbox{7cm}{\begin{center}
  {\large Name Zweitgutachter}\\
  ( )\\
  {\footnotesize Medizinische Fakult\"at\\
	Universit\"at T\"ubingen}\end{center}
 }
  \end{center}
\end{titlepage}

%%%%%%%%%%%%%%%%%%%%%%%%%%%%%%%%%%%%%%%%%%%%%%%%%%%%%%%%%%%%%%%%%%%%%%%%%%%%
%%% Titelr"uckseite: Bibliographische Angaben
%%%%%%%%%%%%%%%%%%%%%%%%%%%%%%%%%%%%%%%%%%%%%%%%%%%%%%%%%%%%%%%%%%%%%%%%%%%%

\thispagestyle{empty}
\vspace*{\fill}
\begin{minipage}{11.2cm}
\textbf{Engel, Sebastian:}\\
\emph{Framework for connecting Trello to other services and applications}\\ Diplom Thesis Informatics\\
Eberhard Karls Universit\"at T\"ubingen\\
Thesis period: 13th march to 11th September 2012
\end{minipage}
\newpage

%%%%%%%%%%%%%%%%%%%%%%%%%%%%%%%%%%%%%%%%%%%%%%%%%%%%%%%%%%%%%%%%%%%%%%%%%%%%

\pagenumbering{roman}
\setcounter{page}{1}

%%%%%%%%%%%%%%%%%%%%%%%%%%%%%%%%%%%%%%%%%%%%%%%%%%%%%%%%%%%%%%%%%%%%%%%%%%%%
%%% Seite I: Zusammenfassug, Danksagung
%%%%%%%%%%%%%%%%%%%%%%%%%%%%%%%%%%%%%%%%%%%%%%%%%%%%%%%%%%%%%%%%%%%%%%%%%%%%


\section*{Abstract}

Trello is a collaboration webservice to manage projects and assign their to-do items to co-workers. There are many collaboration tools available today, but most of them are very basic. Trello however is very extensive and it is optimal fo small businesses. But although it works fine the way it's supposed to it also has its limitations. Trello as it is now is a closed system. Nothing get in or out unless Trello itself is used. But sometimes it would be handy if is would be possible to apply data of Trello to other applications. For example a CMS which should contain alist of completed theses which are already managed in Trello.
 
So this thesis addresses small scripts which let Trello interact with other webservices and applications. To accomplish this task in the most dynamic way possible an API wrapper of the Trello API in the Ruby programming language emerged.

\newpage

\section*{Acknowledgements}
Thanks to Prof. Grust and M.Sc. Tom Schreiber for the oppurtunity to work on this interesting project. A great thank you to my lectors Alena Dausacker and Jennifer Proehl. Thanks to Moritz Uhlig for introducing me to HTML in the first place. Thanks to Sabrina Pfeffer, Martin Lahl, Philipp Wolter, Thomas Zappe and Steffen Zietkowski for the support during my studies. Thanks to my parents for the financial support and the patience.

\cleardoublepage

%%%%%%%%%%%%%%%%%%%%%%%%%%%%%%%%%%%%%%%%%%%%%%%%%%%%%%%%%%%%%%%%%%%%%%%%%%%%%
%%% Table of Contents
%%%%%%%%%%%%%%%%%%%%%%%%%%%%%%%%%%%%%%%%%%%%%%%%%%%%%%%%%%%%%%%%%%%%%%%%%%%%%

\renewcommand{\baselinestretch}{1.3}
\small\normalsize

\tableofcontents

\renewcommand{\baselinestretch}{1}
\small\normalsize

\cleardoublepage

%%%%%%%%%%%%%%%%%%%%%%%%%%%%%%%%%%%%%%%%%%%%%%%%%%%%%%%%%%%%%%%%%%%%%%%%%%%%%
%%% List of Figures
%%%%%%%%%%%%%%%%%%%%%%%%%%%%%%%%%%%%%%%%%%%%%%%%%%%%%%%%%%%%%%%%%%%%%%%%%%%%%

\renewcommand{\baselinestretch}{1.3}
\small\normalsize

\listoffigures

\renewcommand{\baselinestretch}{1}
\small\normalsize

\cleardoublepage

%%%%%%%%%%%%%%%%%%%%%%%%%%%%%%%%%%%%%%%%%%%%%%%%%%%%%%%%%%%%%%%%%%%%%%%%%%%%%
%%% List of tables
%%%%%%%%%%%%%%%%%%%%%%%%%%%%%%%%%%%%%%%%%%%%%%%%%%%%%%%%%%%%%%%%%%%%%%%%%%%%%

\renewcommand{\baselinestretch}{1.3}
\small\normalsize

\lstlistoflistings 

\renewcommand{\baselinestretch}{1}
\small\normalsize

\cleardoublepage

%%%%%%%%%%%%%%%%%%%%%%%%%%%%%%%%%%%%%%%%%%%%%%%%%%%%%%%%%%%%%%%%%%%%%%%%%%%%%
%%% Nomenclature
%%%%%%%%%%%%%%%%%%%%%%%%%%%%%%%%%%%%%%%%%%%%%%%%%%%%%%%%%%%%%%%%%%%%%%%%%%%%%

\renewcommand{\baselinestretch}{1.3}
\small\normalsize

\addcontentsline{toc}{chapter}{Nomenclature}
\printnomenclature[2.5 cm]

\renewcommand{\baselinestretch}{1}
\small\normalsize

\cleardoublepage

%%%%%%%%%%%%%%%%%%%%%%%%%%%%%%%%%%%%%%%%%%%%%%%%%%%%%%%%%%%%%%%%%%%%%%%%%%%%%
%%% Der Haupttext, ab hier mit arabischer Numerierung
%%% Mit \input{dateiname} werden die Datei `dateiname' eingebunden
%%%%%%%%%%%%%%%%%%%%%%%%%%%%%%%%%%%%%%%%%%%%%%%%%%%%%%%%%%%%%%%%%%%%%%%%%%%%%

\pagenumbering{arabic}
\setcounter{page}{1}

%% Introduction
%%%%%%%%%%%%%%%%%%%%%%%%%%%%%%%%%%%%%%%%%%%%%%%%%%%%%%%%%%%%%%%%%%%%
% Introduction
%%%%%%%%%%%%%%%%%%%%%%%%%%%%%%%%%%%%%%%%%%%%%%%%%%%%%%%%%%%%%%%%%%%%
\onehalfspacing
\chapter{Introduction}
\label{Introduction}
Trello is a web service by the New York City based web corporation Fog Creek Software\footnote{Official Fog Creek Software website: \url{http://www.fogcreek.com}}\index{Fog Creek Software}. It is a collaboration tool to manage projects and was launched in 2011\footnote{The original launch post in the Trello blog: \url{http://blog.trello.com/launch/}}. 

\begin{figure}[htb]
\centering
\includegraphics[width=\textwidth]{figures/trello}
\caption{A Trello board.}
\label{fig:trello}
\end{figure}

The service focuses on the concept of so called \emph{boards} containing several configurable lists. Figure \ref{fig:trello} shows a board with the three standard lists \emph{To Do}, \emph{Doing}, and \emph{Done}. In these lists, the user can create to-do items called \emph{cards} containing additional data. Each card has a title and an optional description, some assigned members, a due date, some labels, votes, checklists, comments, and attachments. The creator of the board is the owner and can invite other Trello users to his boards and cards. This way, everyone who's working on a project can see what is going on at the moment. Users who are assigned to a board can even create new to-do items by themselves. If it should be the case that somebody works at more than one company with many projects each, there is the concept of \emph{organizations} to ensure clear separation.

It is possible to manage complete projects within Trello, but sometimes the information in Trello should be displayed in a different manner. For example if there are due dates managed by Trello. It would be useful to display them in the calendar application the user uses otherwise. Trello can't provide information to other applications or web services itself. To work with data outside of Trello, there is a API to access all kinds of data in Trello. The wrapper written for this thesis represents the Trello API in Ruby, and extends it with useful methods to allow developers a convenient possibility to access Trello.

The thesis is organised as follows: The basics are discussed in Chapter \ref{Principles}. This mainly includes a description of the Trello API, a brief introduction to the Ruby programming language and the explanation of the used formats and procedures. Chapter \ref{apiwrapper} deals with the development of an API wrapper in Ruby for the Trello API. Here, the most important methods are described in detail. In the \hyperref[applications]{applications section} the API wrapper is used to link Trello to other services and to extract application-specific data from Trello respectively add new information to Trello. A discussion and a brief outlook on further possible applications of the API wrappers in Chapters \ref{Conclusion} and \ref{Outlook} conclude this thesis.
\cleardoublepage

%% 
%%%%%%%%%%%%%%%%%%%%%%%%%%%%%%%%%%%%%%%%%%%%%%%%%%%%%%%%%%%%%%%%%%%%
% Grundlagen
%%%%%%%%%%%%%%%%%%%%%%%%%%%%%%%%%%%%%%%%%%%%%%%%%%%%%%%%%%%%%%%%%%%%

\chapter{Principles}
  \label{Principles}

\section{Trello}



\begin{figure}[htb]
\centering
\includegraphics[width=\textwidth]{figures/trello-card}
\caption{A opened card in Trello.}
\label{fig:trello-card}
\end{figure}

\subsection{Why Trello}\index{Trello}
Trello is different from most to-do applications. It is streamlined for the purposes of small businesses. It therefore works perfectly for the needs of a university with small groups of people working on the same tasks. Trello has already proved its value for several months. The Trello website is written in HTML 5\index{HTML!5} with the use of AJAX\index{AJAX} where necessary. Trello provides an iOS\index{iOS} \cite{trello:ios} and Android\index{Android} \cite{trello:android} app, both of which are constantly evolving. So the system is state-of-the-art. In addition, the company behind Trello is not just a start-up with three employees, which is also of importance. A product of a small business, which is just based on the enthusiasm of the founders, often doesn't last long. Fog Creek Software is over ten years old and has several products.

The first aim was to see the due dates of the cards someone is assigned to in Google Calendar. Considering this, there were many other use cases for small scripts which could run as cron jobs on a server to serve several regular tasks. These scripts are described in more detail in Chapter 3.

\subsection{Trello API}\index{API}
The Trello API \nomenclature{API}{Application Programming Interface} is still in beta, but it is already very extensive. \cite{trello:docu}

\subsubsection{Authentication}
Because the scripts which are used here need access to private boards in Trello\index{Trello}, there has to be some kind of authentication. For user applications with a front end, the Trello API provides OAuth2. Because of the concept of OAuth2, however, the user is required to enter his Trello username and password. \cite{oauth} The scripts developed for this thesis are supposed to run on servers as cron jobs, so there is no user who could manually enter data. For this kind of applications, Trello provides a key/token-system. Every user has a private key with which they can generate a token. This token will be sent along every request to the Trello API. The token tells Trello which scope the request can access. While generating a token the user can specify the scope of the token and when it will expire. The possible expiration date of a token is between one day and never. In our case we will use \emph{never}. To generate a token the user has to visit a certain URL:
\texttt{
https://trello.com/1/authorize?key=SUBSTITUTEWITHYOURPRIVATEKEY \&name=My+Application\&expiration=never\&response\_type=token \&scope=read,write}
In this example the token would never expire and could read and write everything the user can access with the API. Other valid values instead of \texttt{never} for expiration would be \texttt{1day}, \texttt{30days}. \texttt{30days} is the default value. \cite{trello:gettingstarted}


\section{Ruby}\index{Ruby}
Ruby is a modern general-purpose object-oriented programming language\index{programming language}. The big difference to most other languages is that it focuses on humans rather than computers. 

Yukihiro Matsumoto\index{Matsumoto!Yukihiro}, the designer of Ruby, once said:
\begin{quote}
Ruby is simple in appearance, but is very
complex inside, just like our human body.\cite{ruby:talk}
\end{quote}

That means that Ruby\index{Ruby} is very easy to read and is intuitive\index{intuitive} for humans even though it can perform complex tasks. This is achieved with English keywords instead of brackets and curly brackets. The result of this consistent philosophy is a very easy to read language which is also very plain. Because of the English words instead of abstract characters, Ruby is easy to understand. Even non-programmers are mostly able to understand what is going on. Programmers, as a result, produce a lot fewer errors while writing the code. A wrongly spelt word is more intuitively recognisable than a missing bracket or semicolon. \cite{ruby:about} More about Ruby\index{Ruby} can be found at \url{http://www.ruby-lang.org}.

\subsection{RubyGems}\index{RubyGems}
Ruby has many methods and classes every Ruby installation\index{Ruby!installation} provides\footnote{Libraries that are included with the Ruby standard library\index{library}: \url{http://www.ruby-doc.org/stdlib-1.9.3/}}. In addition, there are hundreds of extensions\index{extensions} for special use cases – to communicate with RESTful web APIs\index{API!RESTful} for example – made by third party developers\index{developer}. In Ruby\index{Ruby}, such extensions\index{extensions} are called \emph{gems}. To manage and publish these third party libraries, there is the standard \emph{RubyGems}\index{RubyGems}. It provides a standard format for third party libraries for Ruby, a tool to manage the installation of them, and a server for distributing the gems. \cite{ruby:gemdev} Some Ruby\index{Ruby} distributions are delivered with several gems. Gems\index{Gem} can be added to an existing Ruby\index{Ruby} installation\index{installation} at any time.

To install an additional gem on a Unix\index{Unix} operating system\index{operating system}, where the Ruby standard library\index{Ruby!standard library} is installed, the following command can be used:  
\begin{center}
\texttt{gem install gemname}
\end{center}
\texttt{gemname} is here referring to the name of the respective gem\index{gem}. If the installation\index{installation} performed without errors, the gem is ready to use. \cite{ruby:gemdoc}

To use an installed gem in a Ruby\index{Ruby} script, the following code at the top of the script before the beginning of the code is necessary:

\begin{lstlisting}[aboveskip=1\baselineskip, caption=Using the gem \emph{gemname}, label=listing001]
require 'gemname'
\end{lstlisting}
Again, \emph{gemname} stands for the name of the respective gem. If any gems that are not part of the Ruby standard library\index{Ruby!standard library} are used in this script, they are listed at the beginning of the related description. Further information at \url{http://doc.rubygems.org} and \url{http://rubyforge.org/projects/rubygems/}. Additionally, \url{http://rubygems.org} is a reference book like website for Ruby gems.

\subsection{Ruby and MySQL}\label{rubymysql}\index{MySQL}\index{Ruby}
In order to access databases\index{database} in Ruby\index{Ruby}, there are additional libraries needed as the Ruby standard library doesn't support any databases. In this API wrapper, the MySQL\index{MySQL} gem is used. This gem is an API wrapper\index{API!wrapper} around the MySQL C client API. MySQL is used here because it is the most popular open source database. \cite{mysql:popularity} It is especially popular for data-handling in web applications. Therefore, the scripts described here will support MySQL\index{MySQL}.

\begin{lstlisting}[aboveskip=1\baselineskip, caption=Initialising a database connection\index{database!connection}., label=listing050]
my = Mysql.init
my.options(Mysql::SET_CHARSET_NAME, 'utf8')
my.real_connect(dbhost, dbuser, dbpassword, db)
my.query("SET NAMES utf8") (*@ \label{line050} @*)	
\end{lstlisting}

Listing \ref{listing050} above shows the initialisation of a new connection to a MySQL\index{MySQL} database with the Ruby MySQL gem. The variable \lstinline{my} is a new database handler. If the correct charset isn't set, it won't work. The host, the user, the password, and the database name should be stored in separate variables so they are easily editable. \lstinline{SET NAMES utf8} in line \ref{line050} of listing \ref{listing050} is a first execution of a MySQL query\index{MySQL!query}. It tells the server the charset name of future connections. In this example, the server expects all future queries as UTF8 \nomenclature{UTF-8}{8-Bit UCS Transformation Format}\nomenclature{UCS}{Universal Character Set} charset.

MySQL statements\index{MySQL!statement} can be executed directly or as \emph{prepared statements}. The example in line \ref{line050} of  \ref{listing050} is a normal statement which is executed directly. Listing \ref{listing051} shows a longer example. 

\begin{lstlisting}[aboveskip=1\baselineskip, caption=Example for a directly executed MySQL query., label=listing051]
response = my.query(" 
	SELECT fieldname1 (*@ \label{line0305} @*)
	FROM tablename 
	WHERE  fieldname2=value
	AND fieldname3=value (*@ \label{line031} @*)
")
response.each do |row|
	puts row (*@ \label{line032} @*)
end
\end{lstlisting}

The MySQL statement\index{MySQL!statement} between line \ref{line0305} and \ref{line031} responds with data, so the result of the query must somehow be read. The response of a \lstinline{SELECT} statement is an array of rows from the given table. To get the lines of the response table, the \lstinline{each} method is used to iterate through the array and print each row.

\begin{lstlisting}[aboveskip=1\baselineskip, caption=\texttt{joomlaMultiple.rb} usage., label=listing029]
statement = my.prepare("
	INSERT INTO tablename (
		fieldname1,
		fieldname2
	) 
	VALUES (
		?,
		?
	)
")
statement.execute value1, value2 (*@ \label{line052} @*)
statement.execute othervalue1, othervalue2 (*@ \label{line053} @*)
\end{lstlisting}

The listing above shows how the MySQL library\index{MySQL!library} for Ruby\index{Ruby} handles prepared statements. The big difference to directly executed statements are the wildcards in the \lstinline{VALUES} part. It's more like a template for a statement. Although the statement doesn't contain all required values, it is sent to the underlying DBMS of MySQL. The DBMS\index{DBMS} can already perform query optimisation\index{query optimisation} on this statement template. To fill in the actual values, the statement must be executed with the \lstinline{execute} method. The lines \ref{line052} and \ref{line053} are two executions of the specified statement. In this step, the wildcards are replaced by the actual values. The DBMS\index{DBMS} performs additional query optimisation\index{query optimisation} if needed. Sometimes, query optimisation depends on the values of the fields. Prepared statements\index{prepared statements} can be executed multiple times. That saves performance, because the initial query optimisation must be performed only once. In addition, prepared statements\index{prepared statements} are safer than directly executed statements. Because the DBMS\index{DBMS} processes the MySQL\index{MySQL} code separately to the actual data, it can distinguish between valid and invalid data. The DBMS checks every given value before writing it to the database. Because of that, SQL injection\index{SQL!injection} isn't possible with prepared statements. For every MySQL statement that writes data to the database\index{database}, it is therefore reasonable to use prepared statements\index{prepared statements}. \cite{mysql:rubydoc}

\subsection{Error handling}\index{error handling}
Because these scripts access mostly web servers, sometimes multiple in a single execution, error handling\index{error handling} is essential. There are many reasons why an API call\index{API!call} can't be executed properly, but the connection is a common reason. It's very useful to the user if  they know the reason why a script failed to execute. Error handling ensures, that the script posts an error message, at least. So the user has a chance to find out what's the reason for the problem and if he can do anything about it.

Error handling in Ruby is realised with \lstinline{begin} blocks. A \lstinline{begin} block is like an \lstinline{if} expression but for \emph{exceptions}\index{exception}. 

\begin{lstlisting}[aboveskip=1\baselineskip, caption=Error handling with \lstinline{begin} blocks., label=listing042]
begin
	my = Mysql.init
	my.real_connect(dbhost, dbuser, dbpassword, db)
rescue => e
	puts e.message
else  (*@ \label{line036} @*)
	puts "It works!"
ensure
	mysql.close if mysql
end
\end{lstlisting}

A \lstinline{begin} block has several sections. The first section is the \lstinline{begin} section. Here, the code to check is executed. The example in listing \ref{listing042} shows a simple database connection. With the \lstinline{begin} block around it, Ruby\index{Ruby} checks if any exceptions\index{exception} occur while executing this code. If there indeed are exceptions, the \lstinline{rescue} section comes to help out. The error is stored in the variable \lstinline{e}. At least the message can be printed here, to inform the user. A \lstinline{begin} block may contain multiple \lstinline{rescue} sections. It's possible to specify special types of exceptions, so corresponding actions can be performed. 

\begin{lstlisting}[aboveskip=1\baselineskip, caption=\texttt{joomlaMultiple.rb} usage., label=listing043]
rescue Mysql::Error => e
\end{lstlisting}

In listing \ref{listing043} the \lstinline{Mysql::Error} class is specified. In this \lstinline{rescue} section only exception\index{exception} of this class will be handled.

An \lstinline{else} section in a \lstinline{begin} block like in line \ref{line036} of listing \ref{listing042}, is allowed only if there are one or more \lstinline{rescue} blocks. It is executed if there are no exceptions. The last section is the \lstinline{ensure} block. It is executed no matter whether there is an exception\index{exception} or not. In the example, the initialised database\index{database} connection is closed. This section is supposed to finish the execution of the whole block. \cite{skorks}

\section{REST}\nomenclature{REST}{Representational State Transfer}
The Trello API is a \emph{RESTful}\index{RESTful} web API. That means that the API is conform to the REST design model. REST is a common style of software architecture for distributed systems. It is built on four of the HTTP request methods\index{HTTP!request method}: GET, POST, PUT and DELETE. An implementation of a RESTful\index{RESTful} web service follows four basic design principles:
\begin{itemize}
	\item Use HTTP\nomenclature{HTTP}{Hyper Text Transfer Protocol} methods explicitly.
	\item Be stateless.
	\item Expose directory structure-like URIs\nomenclature{URI}{Uniform Resource Identifier}.
	\item Transfer XML\nomenclature{XML}{Extensible Markup Language} , JSON\nomenclature{JSON}{JavaScript Object Notation}, or both.
\end{itemize}
\cite{rest}

Following these conventions a GET URL of a RESTful\index{RESTful} web service looks like this:
\begin{center}
\texttt{https://api.trello.com/1/cards/4fc8dd3e1b9ecf0c3571902f? key='PRIVATEKEY\&token=TOKEN}
\end{center}
This is a GET request to get a specific card with the ID \texttt{4fc8dd3e1b9ecf0c3571902f}. If this URL\nomenclature{URL}{Uniform Resource Locator} is visited in a browser\index{browser} (with correct \emph{key} and \emph{token}), the browser will show plain JSON\index{JSON}. In order for Ruby to be able to work with it, it must somehow capture this data. To fulfill the requirements of REST in Ruby, there are several gems. Here the RestClient gem is used. This GET request executed with the RestClient gem in Ruby looks like \ref{listing019}.

\begin{lstlisting}[aboveskip=1\baselineskip, caption=GET request using RestClient., label=listing019]
member = RestClient.get('https://api.trello.com/1/cards/4fc8dd3e1b9ecf0c3571902f?key='+$key+'&token='+$token)
pp JSON.parse(member)
\end{lstlisting}

In comparison to the open-uri library, which is included in the Ruby standard library, using RestClient\index{RestClient} results in much cleaner code, especially when it comes to POST requests.

\begin{lstlisting}[aboveskip=1\baselineskip, caption=POST request using RestClient., label=listing009]
response = RestClient.post(
	'https://api.trello.com/1/boards',
	:name => board['name'], 
	:desc => board['desc'],
	:key => $key,
	:token => $token
)
\end{lstlisting}

\begin{lstlisting}[aboveskip=1\baselineskip, caption=POST request with open-uri., label=listing010]
uri = URI('https://api.trello.com/1/boards')
req = Net::HTTP::Post.new(uri.path)

req.set_form_data(
	'name' => board['name'], 
	'desc' => board['desc'],
	'key'=>$key,
	'token'=>$token
)

Net::HTTP.start(uri.host, uri.port, :use_ssl => uri.scheme == 'https') do |http|
	response = http.request(req)
end
\end{lstlisting}

Listing \ref{listing009} and listing \ref{listing010} show the very same API call. But \ref{listing009} is realised with RestClient and listing \ref{listing010} with open-uri\index{open-uri}. Not only is the open-uri\index{open-uri} code much longer, but open-uri\index{open-uri} also doesn't detect the correct scheme from the given URI. If the call should be performed in HTTPS, this has to be set explicitly. This implies that for the handling of RESTful\index{RESTful} web services, RestClient is the better choice.


\section{JSON}\index{JSON}\label{jsonsec}
All the responses of Trello\index{Trello} API\index{API} calls are in the JSON\footnote{RFC\nomenclature{RFC}{Request For Comments} document for JSON: The application/json Media Type for JavaScript Object Notation (JSON)  \url{http://www.ietf.org/rfc/rfc4627.txt}} format, which is a subset of the JavaScript programming language. Despite its relation to JavaScript, it is language independent. JSON is a data-interchange format like XML, but it is built on two structures. One is a list of key/value pairs. In most programming languages, this is realised as a hash, struct, object, or associative array. The other structure is an ordered list of values. This is realised as an array, list, vector, or sequence in popular programming languages. In JSON itself, these structures are called \emph{object} and \emph{array}. Objects start and end with curly brackets. Each key is followed by a colon and the key/value pairs are separated by commas. Arrays start and end with squared brackets, the values are again separated by commas. Both can be arbitrarily nested. Every time one of the scripts saves content to any other place than Trello, it is also in the JSON\index{JSON} format in order to guarantee easy compatibility with Trello. JSON\index{JSON} can be saved in files, too. A JSON\index{JSON} file has the suffix \texttt{.json}. \cite{json}

\begin{lstlisting}[aboveskip=1\baselineskip, caption=JSON example., label=listing054]
{
    "id": "4eea4ffc91e31d1746000046",
    "name": "Example Board",
    "desc": "This board is used in the API examples",
    "lists": [{
        "id": "4eea4ffc91e31d174600004a",
        "name": "To Do Soon"
    }, {
        "id": "4eea4ffc91e31d174600004b",
        "name": "Doing"
    }, {
        "id": "4eea4ffc91e31d174600004c",
        "name": "Done"
    }]
}
\end{lstlisting}

In listing \ref{listing054}, a JSON example is shown. This is the response of

\begin{center}
\texttt{https://api.trello.com/1/boards/4eea4ffc91e31d1746000046? lists=open\&list\_fields=name,desc\&key=PRIVATEKEY\&token=TOKEN}
\end{center}

The JSON in listing \ref{listing054} starts with a curly bracket. That means the uppermost structure is an object. Here are a few key/value pairs like \texttt{"id": "4eea4ffc91e31d1746000046"}. The key \texttt{"list"} has an array as value, so their value is in squared brackets. Each element of the array is, again, an object.

To process the received JSON, Ruby needs to parse it first. For that purpose, there is the a gem simply called \emph{JSON}. It parses the JSON with the 

\begin{center}
\lstinline{JSON.parse(receivedJson)} 
\end{center}
command, where \lstinline{receivedJson} is a variable that contains the plain JSON received from the Trello API with a GET request. After the parsing, the JSON objects are now represented as Ruby hashes and the JSON arrays as Ruby arrays. To post content to the Trello API, it has to be in JSON\index{JSON}, too. There is another method of the JSON class in Ruby which generates JSON from Ruby hashes and arrays. 

\begin{center}
\lstinline{JSON.generate(hashBoards)} 
\end{center}

The result is a valid JSON formatted string, ready to be written to a JSON file or sent to an API.
\cleardoublepage

%% 
%%%%%%%%%%%%%%%%%%%%%%%%%%%%%%%%%%%%%%%%%%%%%%%%%%%%%%%%%%%%%%%%%%%%
% Trello API wrapper
%%%%%%%%%%%%%%%%%%%%%%%%%%%%%%%%%%%%%%%%%%%%%%%%%%%%%%%%%%%%%%%%%%%%
\onehalfspacing
\chapter{Trello API wrapper}\index{Trello}\index{API}\label{apiwrapper}

These scripts fulfill very different tasks, but they have also much in common. For example, almost every script potentially loads single cards. This API wrapper\index{API!wrapper} makes Trello accessible in Ruby\index{Ruby}. Additionally, the scripts can now use the same methods multiple times and, in consequence, they can stay very lightweight and clean. Almost everything that is possible with the Trello API is also possible with this API wrapper. It doesn't cover all features, though, because the API is still in beta phase, so it changes quite quickly.

\begin{figure}
\centering
\includegraphics[width=\textwidth]{figures/api-wrapper}
\caption{Connections between Trello, the API wrapper, and the actual features. \cite{ruby:icon}\cite{html:logo}\cite{joomla}\cite{google} }
\label{fig: api-wrapper}
\end{figure}

The API wrapper also has methods to pre-process data for Ruby. From a developer's point of view, Trello is all about cards. Cards are the only things in Trello with real data, not just meta data. So if the task is to get a board from the API\index{API}, it means to get the cards of the board. Although the developer won't be able to read out all information, there is an API call to get all cards belonging in a specific board. So the API wrapper has to execute the API call for a single card to cumulate all the information about all the cards of the board. This is the function of the API wrapper, to keep the actual script clean. So the developer can work with the data and doesn't have to worry about determining it.

The API wrapper is meant to perform all API calls which are required by the scripts. None of the API calls should be performed by the scripts that use the API wrapper.


\section{Command Line Interface}\nomenclature{CLI}{Command Line Interface}\index{CLI}\index{Command-Line Interface}\label{cli}
Almost every script needs some information to run properly. The information which every scripts needs, is the key and token of the user whose account is supposed to be used to access Trello. The scripts have to know which cards, lists, and boards they have to look at. Therefore, this information has to be passed on to the scripts, too. This information could be set at the top of each script. But it is obvious that that it would be very unpractical to hard code this in each script. This way, it would be impossible to use the same Ruby file with several Trello accounts. For every Trello account, the user has to generate a dedicated file. The solution for this problem is a command-line interface (CLI)\index{CLI}. With a CLI, the user can pass on information to the script in a predefined format, so the script knows exactly what to do. For every other call, the user can specify different information for the same script.

The Ruby\index{Ruby} class OptionParser\cite{ruby:optionparser}\index{OptionParser} provides easy customisable command-line option analysis. The developer is able to specify their own options for each script. For this purpose, a dedicated class is used. 
In order to let the actual script \emph{know} about the CLI\index{CLI} arguments, the developer has to require the respective CLI class with the command-line option definitions.

\begin{lstlisting}[aboveskip=1\baselineskip, style=bash, caption=Example usage of a script with CLI., label=listing004]
ruby html.rb -c 4ffd78a2c063afeb066408b8
\end{lstlisting}

An example usage of a script with CLI would look like Listing \ref{listing004}. The \texttt{-c} is a command-line option. If there is a string behind the option, like in this case, the string is a so called \emph{argument}. But there are command-line options which stand by themselves. These are called \emph{flags}, which are for polar decisions only.

\begin{lstlisting}[aboveskip=1\baselineskip, caption=Definition of a command-line option, label=listing002]
# Trello list(s)
opts.on("-l", "--lists x,y,z", Array, "Ids of one or more Trello lists.") do |lists| (*@\label{line001}@*)
	options.lists = lists (*@\label{line300}@*)
end
\end{lstlisting}

Listing \ref{listing002} shows the definition of the option \texttt{-l} for passing one or more IDs of lists to a script. In Line \ref{line001} the word \lstinline{Array} casts the list argument to an Array object.

OptionParse provides an automated help option. If the user types 
\begin{center}
\texttt{ruby script.rb -h} 
\end{center}
they get the explanation the developer wrote in the CLI class for this script with all possible options. This list is automatically generated by the definitions of the command-line options like in Listing \ref{listing002}. It works with \texttt{-help} and \texttt{--help} instead of \texttt{-h}, too.

\begin{lstlisting}[aboveskip=1\baselineskip, style=bash, caption=Output of the \texttt{-h} option., label=listing003]
Usage: ical.rb [options]
Select the input cards with -c, -l, -b or -a

Specific options:
 -a, --[no-]all         Set this if all due dates of all cards of all boards this user can see shall be used.
 -l, --lists x,y,z      Ids of one or more Trello lists.
 -b, --boards x,y,z     Ids of one or more Trello boards.
 -o, --organizations x,y,z Ids of one or more Trello organizations.
 -c, --cards x,y,z      Ids of one or more Trello cards.
 -k MANDATORY, --key    Your Trello key.
 -t MANDATORY, --token  The Trello token.
\end{lstlisting}

Listing \ref{listing003} shows the Output of \texttt{ruby ical.rb -h}. 
These are the basic CLI commands used by every script. For some scripts there are additional commands, which are explained in their respective sections.

All information from the command-line is stored in the \texttt{options} variable. This is an instance of Ruby's\index{Ruby} OpenStruct\index{OpenStruct} class, which is a data structure that allows the definition of random attributes. In line \ref{line300} of listing \ref{listing002} the attribute \texttt{lists} is initialised with a new value. \cite{ruby:openstruct} To read the command-line the script has to instruct the command-line class to parse the command-line arguments considering the specified options. After the command-line is parsed the data in available in a OpenStruct:

\begin{lstlisting}[aboveskip=1\baselineskip, caption=Reading the OpenStruct\index{OpenStruct} containing the command-line information., label=listing046]
options = CLHtml.parse(ARGV)

$key = options.key
$token = options.token
\end{lstlisting}

Listing \ref{listing046} shows how to instruct the corresponding command-line class to parse the command-line information and how to access the data in the OpenStruct. Here the token and key specified in the command-line are provided as global variables to the programme.

\section{Methods}
The API wrapper\index{API!wrapper} contains a huge amount of methods. The API\index{API} calls wrapped by the following methods need a private key and a token to access content in private boards. Thus, private key\index{private key} and token\index{token} need not be sent to each method, it will be initialised in each script as global variables. The variable for the private key\index{private key} is called \lstinline{$key} and the one for the token is called \lstinline{$token}. Global variables\index{variable!global} can be accessed from anywhere within the programme during the runtime.

The methods which wrap a \texttt{GET} or \texttt{DELETE} request look like listing \ref{listing045}. A special URL is visited by the RestClient gem and the response JSON is parsed and returned as hash. 

\begin{lstlisting}[aboveskip=1\baselineskip, caption=\lstinline{getBoardsByMember()}, label=listing045]
def getBoardsByMember(memberId)
	boards = RestClient.get("https://api.trello.com/1/members/"+memberId+"/boards?key="+$key+"&token="+$token+"&filter=open")
	boards = JSON.parse(boards)
end
\end{lstlisting}

The methods which wrap \texttt{POST} or \texttt{PUT} requests look like in listing \ref{listing048}. Several arguments have to or can be passed to the URL. 

\vskip 12cm

\begin{lstlisting}[aboveskip=1\baselineskip, caption=\lstinline{postCard()}, label=listing048]
def postCard(cardName, cardDesc, cardPos, idList)
	response = RestClient.post(
		'https://api.trello.com/1/cards',
		'name' => cardName,
		'desc' => cardDesc,
		'pos' => cardPos,
		'idList' => idList,
		'key'=>$key,
		'token'=>$token
	)
	response = JSON.parse(response)
end
\end{lstlisting}

In this case a new card is created. The Trello API needs the title of the new card, the description, if the card should have one, the position in the list, and the ID of the list it which it should be placed. A example JSON response of a \texttt{POST} request to Trello is shown in listing \ref{listing049}.

\begin{lstlisting}[aboveskip=1\baselineskip, style=bash, caption=JSON response of a \texttt{POST} request., label=listing049]
{"id"=>"504bf790e9e68a282c8b3df6",
 "checkItemStates"=>[],
 "closed"=>false,
 "desc"=>"",
 "idBoard"=>"4ffa4c5ce75c29032a88ea31",
 "idChecklists"=>[],
 "idList"=>"4ffa4c5ce75c29032a88ea32",
 "idMembers"=>[],
 "idShort"=>27,
 "manualCoverAttachment"=>false,
 "labels"=>[],
 "name"=>"Woot!",
 "pos"=>3,
 "url"=>"https://trello.com/card/woot/4ffa4c5ce75c29032a88ea31/27",
 "badges"=>
  {"votes"=>0,
   "viewingMemberVoted"=>false,
   "subscribed"=>false,
   "fogbugz"=>"",
   "checkItems"=>0,
   "checkItemsChecked"=>0,
   "comments"=>0,
   "attachments"=>0,
   "description"=>false,
   "due"=>nil}}
\end{lstlisting}
This JSON\index{JSON} response is parsed to a Ruby hash, as well.

For a documentation on every method of the API wrapper there is a documentation in HTML available.

\subsection{Handling of date and time}
Content in Trello may have set dates. These dates are represented by the Trello API as an ISO\nomenclature{ISO}{International Organization for Standardization} 8601\index{ISO!8601}\footnote{More information about ISO 8601 on the website of the International Organization for Standardization: \url{http://www.iso.org/iso/home/store/catalogue_tc/catalogue_detail.htm?csnumber=40874}. The RFC 3339 is a profile of ISO 8601: \url{http://www.ietf.org/rfc/rfc3339.txt}} formatted string. The timezone\index{timezone} of the date is UTC\nomenclature{UTC}{Universal Time Coordinated}. To ensure that the correct time is always displayed, the date has to be adapted to local time. 

\subsubsection{getDate(date, format='de')}
\begin{lstlisting}[aboveskip=1\baselineskip, caption=\lstinline{getDate()}, label=listing044]
def getDate(date, format='de')
	fdate = Time.iso8601(date).getlocal (*@ \label{line045} @*)
	
	if format=='de'
		return fdate.strftime('%d.%m.%Y %H:%M:%S')
	elsif format=='us'
		return fdate.strftime('%m/%d/%Y %I.%M.%S %P')
	elsif format=='joomla'
		return fdate.strftime('%Y-%m-%d %H:%M:%S')
	elsif format=='ical'
		return fdate.strftime('%Y%m%dT%H%M%S')
	elsif format=='year'
		return fdate.strftime('%Y')
	elsif format=='iso8601'
		return fdate.iso8601
	end
end
\end{lstlisting}

In line \ref{line045} of listing \ref{listing044} the given date string in ISO 8601 format is parsed to a Ruby \lstinline{Time} object. \lstinline{getlocal} is the important part here. This function of the \lstinline{Time} class determines the server's time zone and readjusts the time accordingly. This method respects the daylight saving time\index{daylight saving time} in several time zones\index{time zone}, too. For this method working as intended it is important that the correct time zone\index{time zone} is set on the used server. 

The \lstinline{getDate()} method additionally converts the date to other formats. For example date formats which are required by the iCalendar format or the Joomla CMS.

\subsection{Special methods}
Special methods don't just wrap a Trello API\index{API} call, but accumulate basic information with additional data to get information which the Trello API\index{API} doesn't provide directly.

\subsubsection{getCardsAsArray(arrayCardsStd, downloads = true)}
\begin{lstlisting}[aboveskip=1\baselineskip, caption= getCardsAsArray(), label=listing063]
def getCardsAsArray(arrayCardsStd, downloads = true)
	arrayCardsFull = Array.new
	directoryNameAttachments = File.join(Dir.tmpdir, "attachments")
	
	arrayCardsStd.each do |card|
		# export members
		memberArray = Array.new
		card['idMembers'].each do |memberId|
			member = getMember(memberId)
			memberArray << member			
		end
		membersForCard = Hash.new
		membersForCard['members'] = memberArray
		card = card.merge(membersForCard)
		# end export members		
		
		# export checklists
		hasChecklist = getChecklist(card['id']) 
		
		if hasChecklist[0] != nil
			arrayChecklists = Array.new
			checkItemStates = card['checkItemStates']
			hasChecklist.each do |checklist|  
				hashChecklist = Hash.new  
				hashChecklist['id'] = checklist['id']
				hashChecklist['name'] = checklist['name']
				arrayItems = Array.new
				checklist['checkItems'].each do |item|
					hashItem = Hash.new
					hashItem['name'] = item['name']
					hashItem['completed'] = false
					checkItemStates.each do |state|
						if state.value?(item['id'])
							hashItem['completed'] = true
						end
					end
					hashItem['pos'] = item['pos']
					arrayItems.push(hashItem)
				end
				hashChecklist['items'] = arrayItems
				arrayItems = nil
				arrayChecklists.push(hashChecklist)
				hashChecklist = nil
			end
			
			hashCheckListsForCard = Hash.new
			hashCheckListsForCard['checklists'] = arrayChecklists
			
			card = card.merge(hashCheckListsForCard)
		end
		# end export checklists
		
		# export comments
		if card['badges']['comments'] != 0
			comments = getCardComments(card['id'])
			hashCommentsForCard = Hash.new			
			hashCommentsForCard['commentsContent'] = comments			
			card = card.merge(hashCommentsForCard)
		end
		# end export comments
		
		# export attachments
		if card['badges']['attachments'] != 0
			attachments = getAttachment(card['id'])			
			hashAttachmentsForCard = Hash.new			
			hashAttachmentsForCard['attachments'] = attachments			
			card = card.merge(hashAttachmentsForCard)			
			
			if downloads
				# download files
				attachments.each do |attachment|
					fileDomain = URI.parse(attachment['url']).host
					filePath = attachment['url'].gsub(URI.parse(attachment['url']).scheme+"://"+URI.parse(attachment['url']).host, '')
					fileExtension = File.extname(attachment['url'])
					
					fileName = attachment['id']+File.basename(attachment['url'])
					puts "Downloading \'"+fileName+"\'"
								
					if !Dir.exists?(directoryNameAttachments)
						Dir::mkdir(directoryNameAttachments)
					end
					
					Net::HTTP.start(fileDomain) do |http|
							resp = http.get(filePath)
							open(directoryNameAttachments+"/"+fileName, "wb") do |file|
									file.write(resp.body)
							end
					end      
				end
				# download files
			end       
		end	
		# end export attachments
		
		# export votes
		if card['badges']['votes'] > 0
			response = RestClient.get(
					'https://api.trello.com/1/cards/'+card['id']+'/membersVoted?key='+$key+'&token='+$token
			)
			members = JSON.parse(response)
			membersVotedArray = Array.new
			members.each do |member|
				 membersVotedArray.push(member['id'])
			end
			hashMembersVotedForCard = Hash.new			
			hashMembersVotedForCard['membersVoted'] = membersVotedArray
			card = card.merge(hashMembersVotedForCard)	
		end
		# end export votes
		
		arrayCardsFull.push(card)
	end
	
	return arrayCardsFull
end
\end{lstlisting}

The \lstinline{getCardsAsArray()} method is one of the biggest methods of the API wrapper\index{API!wrapper}. The methods to get all cards of a board or a list don't provide all information of the cards. If those functions don't supply all data which is needed for a specific task the missing information has to be added. That is what \lstinline{getCardsAsArray()} is doing. The method is provided with an array of cards. Then it accumulates all information about all cards in this array. That includes members which are assigned to a card, checklists and according check items of a card, comments, attachments and votes. The developer can decide if attachments should be downloaded to a directory in the temporary directory of the operating system. In order to complete that task the method has to iterate through all cards and execute specific API\index{API} calls to get the required information from Trello. Those API calls are wrapped in other methods which are structured like the method in listing \ref{listing045}. The return value of \lstinline{getCardsAsArray()} is an array with the whole information about the required cards.


\subsubsection{isThisMe(memberId)}
\begin{lstlisting}[aboveskip=1\baselineskip, caption= isThisMe(), label=listing059]
def isThisMe(memberId)
	if getMember('me')['id'] == memberId
		return true
	else
		return false
	end
end
\end{lstlisting}

The \lstinline{isThisMe()} method can be used to check if some content of a backup belonged to the actually used Trello account. For example to get all own comments out of an older backup.

\subsection{Accessing CMS}
To access CMS\index{CMS} there are two methods. \lstinline{trelloToJoomlaSingle(joomlaArticleId, articles)} and \lstinline{trelloJoomlaSync(cardId, sectionid, catid, joomlaVersion)}. These are both to write data into the database\index{databse} of the Joomla\index{Joomla} CMS\index{CMS}.


\subsubsection{trelloJoomlaSync(cardId, sectionid, catid, joomlaVersion)}

\begin{lstlisting}[aboveskip=1\baselineskip, caption=Getting standard card information., label=listing029]
card = getSingleCard(cardId)
title = card['name']  (*@ \label{line025}@*)
description = Kramdown::Document.new(card['desc']).to_html  (*@ \label{line026}@*)
\end{lstlisting}

\lstinline{trelloJoomlaSync()} must determine a card's whole information itself. The \lstinline{getSingleCard()} supplies the standard information of a card such as the title in line \ref{line025} of listing \ref{listing029} and the description in line \ref{line026}. Again, the \texttt{kramdown} gem is used to convert the Markdown\index{Markdown} formatted description string to HTML\index{HTML}. But attachments and checklists would also be meaningful to depict in a CMS\index{CMS}. Besides, the cards creation or update date is required to decide whether a card has changed or not.

\vskip 6cm

\begin{lstlisting}[aboveskip=1\baselineskip, caption=Getting the date of a cards last change., label=listing030]
changed = nil
if !cardUpdated(cardId).empty?
	changed = getDate(cardUpdated(cardId).first['date'], 'joomla')
else
	changed = getDate(cardCreated(cardId).first['date'], 'joomla')
end
\end{lstlisting}
There is an API call for the last update of a card. But if the card has never been updated, the response would be empty. So in this case, the API call for the creation date must be used.

\begin{lstlisting}[aboveskip=1\baselineskip, caption=Processing the attachments of a card., label=listing031]
hasAttachment = getAttachment(cardId) (*@ \label{line027}@*)

if hasAttachment[0] != nil
	description += "<ul>"		
	hasAttachment.each do |att|	
		description += "<li><a href=\""+att['url']+"\">\""+att['name']+"\"</a></li>"
	end
	description += "</ul>"
end
\end{lstlisting}

To get the attachments of a card, the \lstinline{getAttachment(cardId)} method is used in line \ref{line027} of listing \ref{listing031}. If the API\index{API} call contains attachment data, the HTML\index{HTML} tag \lstinline{<ul>} is appended to the description. After that, the attachments are appended as list items. Their names are simply linked with the URL\index{URL} to the file on Trello's servers. 

\begin{lstlisting}[aboveskip=1\baselineskip, caption=Processing the checklists of a card., label=listing032]
hasChecklist = getChecklist(cardId) 

if hasChecklist[0] != nil
	hasChecklist.each do |checklist| 			
		description += "<h4>"+checklist['name']+"</h4>"
		description += "<ul>"
		checklist['checkItems'].each do |item|	
			if isCompleted(cardId, item['id'])
				description += "<li><del>"+item['name']+"</del></li>"
			else
				description += "<li>"+item['name']+"</li>"
			end
		end
		description += "</ul>"
	end	
end
\end{lstlisting}

To process the checklists of a card, the method \lstinline{getChecklist(cardId)} is called first. If the response contains checklist data, a \lstinline{<h4>} HTML tag with the checklist's name is added to the description. After that, another \lstinline{<ul>} is started. The \lstinline{isCompleted()} method must be called for every checklist item to resolve its status. In dependency of the result, the name of the item is displayed as crossed out or not. Ruby's\index{Ruby} append method is used because the description already is a HTML\index{HTML} string.

Now, that the content is available to the script, it must be written to the database. Because the \lstinline{trelloJoomlaSync()} method supports Joomla\index{Joomla} versions 1.5 and 2.5, every database query exists twice. From Joomla 1.5 to Joomla\index{Joomla} 2.5, the underlying database\index{database} structure has changed a bit. 

\begin{lstlisting}[aboveskip=1\baselineskip, caption=\texttt{joomlaMultiple.rb} usage., label=listing033]
begin  
	existingArticleQuery = my.query(" (*@ \label{line028}@*)
		SELECT id, created, modified
		FROM jos_content 
		WHERE metadata='"+cardId+"'
	") (*@ \label{line029}@*)
rescue Mysql::Error => e
	puts e
else
	# if article doesn't exist insert it into the db
	if existingArticleQuery.num_rows == 0
		begin
			
			# Insert new article, see listing (*@ \ref{listing034} @*)(*@ \label{line030}@*)
			
		rescue Mysql::Error => e
			puts e
			return
		ensure
			stmt.close if stmt
		end			
	else
		# this should be only one because per Trello card ID should only exist one article in Joomla
		existingArticleQuery.each do |thisArticle|
			
			existingId = thisArticle[0]
			existingCreated = thisArticle[1]
			existingModified = thisArticle[2]
			
			# check if the modified timestamp im Trello is different to the modified timestamp in Joomla
			begin 
				if existingModified != changed
					
					# Update article, see listing (*@ \ref{listing036} @*)(*@ \label{line032}@*)
					
					puts 'Changed: '+cardId+" : "+title
				else 
					puts 'Nothing changed: '+cardId+" : "+title
				end					
			rescue Mysql::Error => e
				puts e
				return
			ensure
				stmt.close if stmt
			end
		end
	end	
ensure
	my.close if my
end
\end{lstlisting}

In the first \lstinline{begin} block of listing \ref{listing033}, from line \ref{line028} to line \ref{line029}, the method scans the database table \lstinline{jos_content} for an article with the actual handled card ID in the \lstinline{metadata} field. If the response of this query is an empty array, the card isn't in the database. The script then must insert a new article into the database\index{database}. This is performed in line \ref{line030}. The corresponding MySQL\index{MySQL} code is showed in listing \ref{listing034}. If the resulting array in line \ref{line028} contains a row, the script must check if the new data is more recent than the data in the database. This array cannot contain more than one row because the script inserts the article just once if it is new, otherwise it replaces the old article with an updated version. In order to do that, the script looks at the \lstinline{modified} field of the existing article. This date is saved in the \lstinline{existingModified} variable. The update date of the actual Trello card, which is determined in listing \ref{listing030}, is stored in the \lstinline{changed} variable. If \lstinline{existingModified} differs from \lstinline{changed}, the article is updated with the new date of the Trello card in line \ref{line032}. The script assumes that the Trello card always contains correct data. So it's not necessary to check back if the \lstinline{changed} is actually more recent. The MySQL statement\index{MySQL!statement} of the update statement is showed in listing \ref{listing036}.

\begin{lstlisting}[aboveskip=1\baselineskip, caption=Insert new article in the Joomla database., label=listing034]
begin
	stmt = my.prepare("
		INSERT INTO jos_content (
			title, 
			alias, 
			`introtext`, 
			state, 
			sectionid, 
			catid, 
			created, 
			created_by, 
			modified,
			parentid, 
			ordering, 
			access,					
			metadata
		)
		VALUES (
			?, 
			?, 
			?, 
			1, 
			?, 
			?, 
			?, 
			62, 
			?,
			0, 
			1, 
			0,
			?
		)
	")
	
	stmt.execute title, title.downcase, description.gsub((*@/'/@*), '&(*@\#@*)39;'), sectionid, catid, changed, changed, cardId
	puts 'New article: '+cardId+" : "+title
rescue Mysql::Error => e
	puts e
	return
ensure
	stmt.close if stmt
end
\end{lstlisting}


\begin{lstlisting}[aboveskip=1\baselineskip, caption=Updating existing Joomla article., label=listing036]
stmt = my.prepare("
	UPDATE jos_content 
	SET
		title = '"+title+"',
		alias = '"+title.downcase+"',
		`introtext` = '"+description.gsub(/'/, '&#39;')+"',
		state = 1,
		sectionid = 5,
		catid = 34,
		created = '"+changed+"',
		created_by = 62,
		modified = '"+changed+"',
		parentid = 0,
		ordering = 1,
		access = 0
	WHERE
		metadata = '"+cardId+"'
")
stmt.execute
\end{lstlisting}

\subsubsection{trelloToJoomlaSingle(joomlaArticleId, articles)}

\lstinline{trelloToJoomlaSingle()} works like \lstinline{trelloJoomlaSync()}, but it writes all information in Joomla\index{Joomla} in one single article and in HTML\index{HTML} format.

\begin{lstlisting}[aboveskip=1\baselineskip, caption= trelloToJoomlaSingle(), label=listing061]
def trelloToJoomlaSingle(joomlaArticleId, articles)
	# Database connection
	dbhost = 'host'
	dbuser = 'user'
	dbpassword = 'password'
	db = 'dbname'
	
	htmlSite = "<h3>Universit&auml;t T&uuml;bingen</h3>"
	
	htmlSite << "<p> </p>
	<table style=\"text-align: center;\" border=\"0\">
	<tbody>
	<tr style=\"background-color: #c3d2e5;\">
	<td style=\"text-align: center;\">
	<p style=\"text-align: left; padding-left: 5px;\"><strong><span><strong> Thema</strong></span></strong></p>
	</td>
	</tr>"
	
	i = 0
	articles.each do |element|
		title = element.title
		description = element.description
		if element.attachments != []		
			attachments = element.attachments
		end
		
		htmlSite << "
		<tr style=\"background-color: "
		if i.even? 
			htmlSite << "#e0e8ec;"
		else
			htmlSite <<"#c3d2e5"
		end
		htmlSite << "\">
		<td><p style=\"text-align: left; padding-left: 5px;\"><span><strong>"
		htmlSite << title
		htmlSite << "</strong></span></p>
		<div style=\"text-align: left; padding-left: 5px;\"><span style=\"font-size: xx-small;\">"
		htmlSite << description		
		htmlSite << "</span></div>
		<div style=\"text-align: left;\"><span style=\"font-weight: normal; font-size: small;\"> 
		<ul>"
		if element.attachments != []
			attachments.each do |attachment|
				name = attachment.name
				url = attachment.url
				htmlSite << "<li><a href=\""
				htmlSite << url
				htmlSite << "\">"
				htmlSite << name
				htmlSite << "<a/></li>"
			end	
		end	
		htmlSite << "</ul>
		</span></div>
		</td>
		</tr>"	
		i += 1
	end
	i = nil
	
	htmlSite << "</tbody>
	</table>"
	
	#save to file	
	fileHtml = File.new("arbeiten.html.tmp", "w+")
	fileHtml.puts "<!doctype html>
	<head>
		<meta charset=\"UTF-8\">
		<meta http-equiv=\"X-UA-Compatible\" content=\"IE=edge,chrome=1\">
	
		<title>Abgeschlossene Arbeiten</title>
	</head>
	<body>"
	fileHtml.puts htmlSite	
	fileHtml.puts "</body></html>"
	File.rename("arbeiten.html.tmp", "arbeiten.html")
	
	#save to DB
	my = Mysql.init
	my.options(Mysql::SET_CHARSET_NAME, 'utf8')
	my.real_connect(dbhost, dbuser, dbpassword, db)
	my.query("SET NAMES utf8")

	stmt = my.prepare("UPDATE jos_content SET `introtext`='"+htmlSite+"' WHERE id="+joomlaArticleId.to_s)
	stmt.execute
	
	my.close if my
	
end
\end{lstlisting}











\cleardoublepage

%%
%%%%%%%%%%%%%%%%%%%%%%%%%%%%%%%%%%%%%%%%%%%%%%%%%%%%%%%%%%%%%%%%%%%%
% Diskussion und Ausblick
%%%%%%%%%%%%%%%%%%%%%%%%%%%%%%%%%%%%%%%%%%%%%%%%%%%%%%%%%%%%%%%%%%%%

\chapter{Applications}
  \label{Applications}

\section{Trello API wrapper}

\section{Trello framework}

\section{Export to HTML}

\subsection{Twitter Bootstrap Framework}

\subsection{HTML 5}

\subsection{CSS 3 / SASS}

\subsection{ERB / Templating}

\section{One way sny to Google Calendar}

\section{Export to iCal}

\section{One way sync to Joomla}

\subsection{For every card an article}

\subsection{All cards in one article}

\subsection{One way sny to WordPress}

\section{Backup}

\subsection{Export}

\subsection{Import}

\subsubsection{Filename option}
The -n (or -name) argument for this script stands for the filename of the backup file which contains the  exported Trello data. With -n the user can specify a file to import. While processing the script first checks if the user has passed this argument. If not, it aborts. If the -n argument is given, the scipt proofes if the file is a ZIP file. For that it soesn't use the filename but the MIME type of the file.

\begin{lstlisting}[float=htb, caption=Bewegungsdaten auslesen \cite{apple:003}, label=listing008]
if `file -Ib #{@filename}`.gsub(/;.*\n/, "") != "application/zip"
	puts "ERROR: The backup file has to be a ZIP file!"
	abort
end
\end{lstlisting}

	
In line 1 the \texttt{file -Ib \#\{\@filename\}} is a bash call for receiving the MIME type of a file. Ruby executes it and with the gsub-Method it cuts the MIME part out of the received string.

\subsection{Member import}


\cleardoublepage

%%
%%%%%%%%%%%%%%%%%%%%%%%%%%%%%%%%%%%%%%%%%%%%%%%%%%%%%%%%%%%%%%%%%%%%
% Diskussion und Ausblick
%%%%%%%%%%%%%%%%%%%%%%%%%%%%%%%%%%%%%%%%%%%%%%%%%%%%%%%%%%%%%%%%%%%%

\chapter{Conclusion}
  \label{Conclusion}

\todo{Conclusion?}
\cleardoublepage

%%
%%%%%%%%%%%%%%%%%%%%%%%%%%%%%%%%%%%%%%%%%%%%%%%%%%%%%%%%%%%%%%%%%%%%
% Diskussion und Ausblick
%%%%%%%%%%%%%%%%%%%%%%%%%%%%%%%%%%%%%%%%%%%%%%%%%%%%%%%%%%%%%%%%%%%%

\chapter{Outlook}
  \label{Outlook}

\section{Trello Alfred Extension}

Alfred \cite{alfred} is a small Mac application which simplifies the way one can search the web or access all sorts of applications. It constist just of a input field which one cann access with a keystroke combination. It's like an extended Spotlight (on Mac) or Windows Search (on Windows). Developers can write extensions to access other webservices and applications with Alfred. It's even possible to run scripts with Alfred. With that possibility given it's perfect for acessing Trello while working in a fast and easy way. 



There are three commands to add or read cards with this extension:

\begin{enumerate}
	\item \texttt{trello board-name} will return the card-names and statuses of this board.
	\item \texttt{trello board-name list-name} will return card-names and statuses of this list in this board.
	\item \texttt{trello board-name text for a new card} will add a new card with the specified text to the first list of this board.
	\item \texttt{trello board-name list-name text for a new card} will add a new card with the specified text to this list of this board.
\end{enumerate}

If you enter \texttt{trello Berlin Visit the Reichstag} in Alfred the extension looks for a board called \emph{Berlin}. If it finds nothign it looks for \emph{Berlin Visit} and so on. So your board names shouldn't end with an imperative. The thought behind this operating principle is that it's very unlikely that a board name ends with an imperative and that imperatives are often used for card titles because cards are sort of a command.

\begin{figure}[htb]
\centering
\includegraphics[width=\textwidth]{figures/trello-ext}
\caption{Alfred Extension for Trello: This command would add a card with the name \emph{Visit the Reichstag} to the board called \emph{Berlin}.}
\label{fig:trello-ext}
\end{figure}

If you omit the text after the board name the extension will show you all card names of this board and its statuses.

Sometimes there are several boards with similar board names. In this case the extension will pick the \textquotedblleft last\textquotedblright match. So if you have two boards called \emph{Berlin} and \emph{Berlin sightseeing} the extension will would pick \emph{Berlin sightseeing}. This approach makes sense because if the extension would pick the first match, in this case \emph{Berlin}, it wouldn't be possible to access \emph{Berlin sightseeing}. In the case that one wants to access \emph{Berlin} and add a new card beginning with \emph{sightseeing} one has to put this board name betweet tick marks.

\todo{Code this and verify the practicability.}

\section{Native applications}
Although Trello is an extremely good web-app, I'm of the opinion that a native application is always the better solution. The first reason is because it's a dedicated app and so it's integrated with the operationg system. Especially for todo-applications it's an advantage that they can access the systems notification system, or that they could completely vanish in the background so they don't bother the user while working. There are mobile applications for iOS \cite{trello:ios} and Android \cite{trello:android} by Trello itself. But theres no Mac, Windows or Linux application.

A native application would even speed up the Alfred extension because the application could cache the data. So there hasn't to be an actual HTTP request for every command by the Alfred extension. And if a HTTP request necessary the user hasn't to wait because the application will handle the command in the background.
\cleardoublepage


%%%%%%%%%%%%%%%%%%%%%%%%%%%%%%%%%%%%%%%%%%%%%%%%%%%%%%%%%%%%%%%%%%%%%%%%%%%%%
%%% Bibliographie
%%%%%%%%%%%%%%%%%%%%%%%%%%%%%%%%%%%%%%%%%%%%%%%%%%%%%%%%%%%%%%%%%%%%%%%%%%%%%

\addcontentsline{toc}{chapter}{Bibliography}

\bibliographystyle{alpha}
\bibliography{sources.bib}
%% Obige Anweisung legt fest, dass BibTeX-Datei `sources.bib' verwendet
%% wird. Hier koennen mehrere Dateinamen mit Kommata getrennt aufgelistet
%% werden.

\cleardoublepage

%%%%%%%%%%%%%%%%%%%%%%%%%%%%%%%%%%%%%%%%%%%%%%%%%%%%%%%%%%%%%%%%%%%%%%%%%%%%%
%%% Index
%%%%%%%%%%%%%%%%%%%%%%%%%%%%%%%%%%%%%%%%%%%%%%%%%%%%%%%%%%%%%%%%%%%%%%%%%%%%%

\addcontentsline{toc}{section}{Index} 
\printindex

\cleardoublepage

%%%%%%%%%%%%%%%%%%%%%%%%%%%%%%%%%%%%%%%%%%%%%%%%%%%%%%%%%%%%%%%%%%%%%%%%%%%%%
%%% Erklaerung
%%%%%%%%%%%%%%%%%%%%%%%%%%%%%%%%%%%%%%%%%%%%%%%%%%%%%%%%%%%%%%%%%%%%%%%%%%%%%
\thispagestyle{empty}
\section*{Statement of authorship}

I certify that I have prepared this thesis independently and that I'm using only the tools mentioned here. All passages that have been taken from other works are characterized as borrowing. This thesis has been submitted in identical or similar form in any other program as an examination.

\vskip 3cm

Place, Date	\hfill Signature \hfill 
%%%%%%%%%%%%%%%%%%%%%%%%%%%%%%%%%%%%%%%%%%%%%%%%%%%%%%%%%%%%%%%%%%%%%%%%%%%%%
%%% Ende
%%%%%%%%%%%%%%%%%%%%%%%%%%%%%%%%%%%%%%%%%%%%%%%%%%%%%%%%%%%%%%%%%%%%%%%%%%%%%

\end{document}

